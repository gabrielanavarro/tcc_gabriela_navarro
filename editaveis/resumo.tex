\begin{resumo}
A criptografia tem duas formas principais para cifrar um texto utilizando chave simétrica e chave assimétrica. Os algoritmos que utilizam chave simétrica são divididos em algoritmos de cifra de bloco e cifra de fluxo. Uma falha muito explorada por atacantes que desejam quebrar um texto cifrado é da análise de frequência de ocorrência dos caracteres do mesmo, pois a frequência média de ocorrência é conhecida para cada língua e muitos dos algoritmos não se preocupam com o balanceamento dessa frequência enquanto cifrando o texto. 
Este trabalho de conclusão de curso apresenta uma proposta de algoritmo que irá realizar o balanceamento completo da frequência de caracteres, aumentando sua segurança contra atacantes e curiosos.

 \vspace{\onelineskip}
    
 \noindent
 \textbf{Palavras-chaves}: criptografia. cifra de fluxo. balanceamento.
\end{resumo}
