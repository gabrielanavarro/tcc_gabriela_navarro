\lstinputlisting{ ../algoritmo/cipher.c }
\lstinputlisting{ ../algoritmo/decipher.c }
\lstinputlisting{ ../algoritmo/functions.c }
\lstinputlisting{ ../algoritmo/rc4Decipher.c }
\lstinputlisting{ ../algoritmo/functions.h }
\lstinputlisting{ ../algoritmo/test.c }
\lstinputlisting{ ../algoritmo/rc4Cipher.c }
