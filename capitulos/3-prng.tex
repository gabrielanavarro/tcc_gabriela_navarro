\chapter{Pseudo Random Number Generator}
\label{pseudo-random-number-generator}

% 
Há muitos anos a geração de números aleatórios é objeto de estudo da matemática. Por muito tempo não tinha um algoritmo determinístico para a geração de números que tivesse um período de repetição suficientemente longa para aplicações complexas. Em 1998, os matemáticos japoneses Matishumote e Nishimura inventaram o primeiro algoritmo em que a periocidade ( $2 ^ {19937}$ - 1) excede o número de mudanças rotatórias de elétrons desde a criação do universo. ~\cite{cristophe-diethelm} 

%
Apesar de existir o \textit{TRNG} ou \textit{True Random Number Generator}, que oferece sequências de números que são aleatórios, as mesmas não são determinísticas e isso faz com que sua utilização seja limitada, visto que em aplicações como a criptografia é necessário reproduzir a mesma sequência para a função de cifrar e de decifrar. 

%
Com isso a utilização de geradores de números pseudo aleatórios se faz necessária, porém a sequência gerada deve seguir dois critérios para que seus números possam ser considerados aleatórios:

\begin{description}
	\item [Aleatoriedade]
	O principal requisito para uma sequência ser considerada aleatória é seus números serem aleatórios. Dois critérios podem ser usados para validar uma sequência:
		\begin{description}
			\item [Distribuição uniforme]
			Os bits de uma sequência devem ter frequência aproximada, ou seja, a quantidade de 0 deve ser aproximada da quantidade de 1.
			\item [Indepêndência]
			Os números não devem ter relação entre si.
			
		\end{description}
	\item [Imprevisibilidade]
	Esse critério serve para garantir que sabendo um número da sequência, é impossível prever o número anterior ou posterior.
\end{description}

Um gerador de números pseudo-aleatórios tem como entrada uma semente \footnote{Termo para especificar um número para geração de uma sequência.} A semente normalmente é gerada de um algoritmo \textit{TRNG}. Pelo fato de serem utilizadas como entrada em algoritmos determinísticos as sementes devem ser guardadas de forma segura.

%\begin{figure}[h]
%	\centering
%	\includegraphics[scale=1]
%	{figuras/prng.eps}
%	\caption[Esquema de um \textit{PRNG}]{Esquema de um \textit{PRNG}\protect\footnotemark}
%\end{figure}
%\footnote{Fonte: \url{www.pit-claudel.fr/clement/blog/how-random-is-pseudo-random-testing-pseudo-random-number -generators-and-measuring-randomness/}}


\section{Geradores}
\subsection{Linear Congruential Generator}

Uma técnica muito usada para geração de números pseudo-aleatórios é um algoritmo que primeiramente foi proposto por Lehmer, que é conhecido como método \textit{Linear Congruential}. \cite{william-stallings} O algoritmo gera uma sequência dependendo das entradas e a função utilizada para essa geração é:
\begin{equation}
	\label{Função para geração de números aleatórios.}
	\mathcal{X}_n = (a \times X_{n - 1} + c)\: \bmod \: n
\end{equation}

Para cada número na sequência, essa formula é aplicada. A eficiência dessa sequência está diretamente relacionada a escolha das variáveis. A variável \textbf{m} é o módulo da equação e condição de escolha dessa variável é ser maior que 0. Para a variável \textbf{a} é o multiplicador da função e sua condição é ser maior que 0 e menor que \textbf{m}. A variável c é o incremento, sua condição é é estar entre 0 e \textbf{m}. O \textbf{$X_0$} é a semente para a sequência, será o primeiro elemento da sequência e sua condição é estar entre 0 e \textbf{m}.

A escolha dessas entradas determina a periodicidade da sequência e por consequência seu desempenho. Como exemplo de uma sequência com periodicidade 4, as entradas seriam:

\begin{table}[h]
	\centering
	\begin{tabular}{|l|l|}	
	\hline
		Variável & Valor \\ \hline
		\textbf{a} & 5 \\ \hline
		\textbf{c} & 0 \\ \hline
		\textbf{m} & 16 \\ \hline
		\textbf{$X_0$} & 1 \\ \hline
	\end{tabular}
	\caption{Parâmetros para sequência com periodicidade 4}
\end{table}

Portanto o primeiro valor da sequência é o 1, por ser a semente. O segundo valor é calculado da seguinte maneira: 

\begin{equation}
	\label{Cálculo do segundo valor}
	\mathcal{X}_n = ( 5 \: \times \: 1 \: + \: 0) \: \bmod \: 16
\end{equation}

\begin{equation}
	\label{Resultado do segundo valor}
	\mathcal{X}_n = 5
\end{equation}

Para o terceiro valor, o cálculo é feito utilizando o resultado do anterior:

\begin{equation}
	\label{Cálculo do terceiro valor}
	\mathcal{X}_n = ( 5 \: \times \: 5\: +\: 0)\: \bmod \: 16
\end{equation}

\begin{equation}
	\label{Resultado do terceiro valor}
	\mathcal{X}_n = 9
\end{equation}

Finalmente para o quarto valor, também se utiliza o valor anterior em seu cálculo:

\begin{equation}
	\label{Cálculo do quarto valor}
	\mathcal{X}_n = ( 5 \: \times \: 9 \: + \: 0) \: \bmod \: 16
\end{equation}

\begin{equation}
	\label{Resultado do quarto valor}
	\mathcal{X}_n = 13
\end{equation}

A partir do quinto valor, a sequência se repete:

\begin{equation}
	\label{Cálculo do quinto valor}
	\mathcal{X}_n = ( 5 \: \times \: 13 \: + \: 0) \: \bmod \: 16
\end{equation}

\begin{equation}
	\label{Resultado do quinto valor}
	\mathcal{X}_n = 1
\end{equation}

Ou seja, as escolhas dos parâmetros é um passo extremamente importante para esse gerador. Um exemplo com uma periodicidade relativamente grande pode ser obtida trocando apenas o valor de \textbf{m} da resolução anterior:

\begin{table}[h]
	\centering
	\begin{tabular}{|l|l|}	
		\hline
		Variável & Valor \\ \hline
		\textbf{a} & 5 \\ \hline
		\textbf{c} & 0 \\ \hline
		\textbf{m} & 37 \\ \hline
		\textbf{$X_0$} & 1 \\ \hline
	\end{tabular}
	\caption{Parâmetros substituindo o valor de \textbf{m}}
\end{table}

A sequência gerada a partir dessas variáveis, é:

[1, 5, 25, 14, 33, 17, 11, 18, 16, 6, 30, 2, 10, 13, 28, 29, 34, 22, 36, 32, 12, 23, 4, 20, 26, 19, 21, 31, 7, 25, 24, 9, 8, 3, 15]

Apesar de ter sido maior que a sequência anterior, isso não necessariamente prova sua eficiência, visto que para aplicações como a criptografia, números muito maiores seriam utilizados e sua periodicidade deve ser maior também.


\subsection{Blum Blum Shub Generator}
Uma estratégia para se gerar números pseudo-aleatórios seguramente é conhecido como Blum Blum Shub, nomeado pelos seus desenvolvedores. Tem talvez a prova pública de sua força criptográfica de algoritmos para qualquer propósito. ~\cite{william-stallings} Os primeiros passos para o algoritmo são:

\begin{enumerate}
	\item Escolha números primos \textbf{p} e \textbf{q}.
	\item \textbf{p} e \textbf{q} devem ter resto 4 quando divididos por 4.
	\item Obtenha \textbf{n}, multiplicando \textbf{p} e \textbf{q}.
	\item Escolha \textbf{s} tal que \textbf{s} é primo relativo de \textbf{n}\footnote{p e q não são fatores de s}.
\end{enumerate}

Com isso o gerador produzirá uma sequência com as seguintes condições:

\begin{equation}
	\label{Equação para produzir a sequência}
	\mathcal{X}_0 = s^2\: \bmod \: n
\end{equation}

\begin{equation}
	\label{Equação para produzir a sequência}
	\mathcal{X}_n = (X_{n-1})^2 \: \bmod \: n
\end{equation}

\begin{equation}
	\label{Equação para produzir a sequência}
	\mathcal{B}_n = X_n \: \bmod \: 2
\end{equation}
Ou seja, os bits menos significantes são retirados a cada número produzido.

\subsection{Gerador Blum Micali}
\label{blum-micali-generator}

O gerador Blum Micali é um gerador de números pseudo-aleatórios provadamente seguro e foi criado por Manuel Blum, que também participou do desenvolvimento do Blum Blum Shub, e Silvio Micali. ~\cite{josefin-martin}

O gerador é executado utilizando os seguintes passos iniciais:

\begin{enumerate}
	\item Escolha um número primo \textbf{p}
	\item Então o grupo cíclico\footnote{Grupo que é gerado por um elemento que e que o grupo finaliza nesse mesmo elemento.} Z$_p$ = [1, p-1]
	\item a é um elemento de Z$_p$ e é o número aleatório resultante
\end{enumerate}


Com esses passos, o algoritmo tem as entradas para que possa ser realizada a geração dos números. A função que de geração é:

\begin{equation}
	\mathcal{G}^k \equiv a \bmod p
\end{equation}

Em que \textbf{k} é um número inteiro postivo e g um inteiro que representará um grupo de números inteiros aleatórios.

\subsection{Exemplo}

Dado que:

\begin{enumerate}
	\item p = 17
	\item Z$_p$ = [1,16]
\end{enumerate}

Então para g = 3 e k = 4: 3$^4$ $\equiv$ 13 $\bmod$ 17. Ou seja, o número aleatório 13 satisfaz a condição de estar incluso no grupo Z$_p$ e é o número aleatório do grupo 3.

\section{Segurança comprovada em geradores de números pseudo-aleatórios}
\label{security-of-prng}

Segurança comprovada em criptografia significa que quebrar um algoritmo critpográfico específico é tão difícil quanto resolver um problema específico conhecido por ser difícil. ~\cite{josefin-martin}

Uma das aplicações de números pseudo aleatórios seguros é servir de geradores de \textit{keystream} para a cifra de fluxo. O algoritmo nesse trabalho proposto, irá utilizar esse de tipo de gerador de \textit{keystream}.

Existem dois geradores de números pseudo-aleatórios que são publicamente comprovados em segurança: \textit{Blum Blum Shub} e \textit{Blum Micali}.

\subsection{Segurança do Blum Blum Shub}

A segurança do \textit{Blum Blum Shub} é obtida pelo fato de que seu funcionamento se baseia na residuasidade quadrática. Esse problema é relacionado em decidir se \textbf{x}, em que mdc(x,N) = 1 e que \textbf{x} está entre 1 $\le$  x $\le$ N. \footnote{N é igual a multiplicação dos números primos escolhidos para a geração dos números.}

Para ser um resíduo quadrático, \textbf{x} tem que ter a seguinte propriedade $y ^ 2 \equiv x \: \bmod N$, para um inteiro y. Para esse cálculo, um algoritmo recebe N e x como parâmetros e retorna 1 se for resíduo quadrático e 0 se não for.~\cite{josefin-martin}

\subsection{Segurança do Blum Micali}

Essa segurança é provida pelo problema do logaritmo discreto. Esse gerador utiliza da seguinte função:

\begin{equation}
	\mathcal{G}^k \equiv a \: \bmod \: p 
\end{equation}

Realizar a exponenciação logarítmica é um problema fácil de resolver, porém o inverso, o logarítmo discreto, é uma operação comprovadamente difícil.

\subsection{Comparação do Blum Blum Shub E Blum Micali}

No trabalho de Josefin Agerblad e  Martin Andersen em \textit{Provably Secure Pseudo-Random Generators}, foi realizado uma comparação dos dois geradores de números pseudo aleatórios que são comprovadamente seguros nos quesitos de aplicação, velocidade e segurança.

No quesito aplicação, o \textit{Blum Blum Shub} foi considerado mais vantajoso pelo fato que esse gerador é melhor aplicacado para cifração de chave público e o \textit{Blum Micali} ser mais vantajoso para cifrações de chave privada. Como a cifração de chave pública é mais utilizada na criptografia, o \textit{Blum Blum Shub} foi ganhador nesse quesito.

No quesito velocidade, os autores não conseguiram determinar qual gerador é mais rápido, visto que para um gerador ser considerado seguro, o desempenho do mesmo é prejudicado já que suas operações são mais complexas.

No quesito segurança, os autores também não conseguiram chegar em um consenso, visto que o problema que dificulta a quebra dos dois geradores são distintos e igualmente difíceis de serem resolvidos.