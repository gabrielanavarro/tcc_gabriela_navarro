\chapter{Conclusão}
\label{conclusion}

A proposta desse trabalho era fazer a definição de um algoritmo de cifra de fluxo com balanceamento de saída.

Existem diversas técnicas de criptoanálise e uma das mais antigas e que se aplicam a qualquer tipo de algoritmo de criptografia é a análise de frequência de um texto cifrado. Essa proposta de algoritmo irá dificultar esse método, visto que todas as frequências no texto cifrado serão as mesmas.

\section{Trabalhos Futuros}
\label{future-work}

	\subsection{Implementação}
	\label{implementation}
	
	O primeiro passo para os próximos trabalhos é a implementação do algoritmo. Para esse trabalho, o mesmo será realizado em linguagem C, tomando os devidos cuidados na alocação de memória. 
	A implementação de vários geradores de números pseudo-aleatórios também será realizada e de alguns algoritmos como por exemplo o RC4.
	
	\subsection{Testes}
	\label{tests}
	
	Um problema identificado no decorrer da definição do algoritmo foi a possível queda de desempenho na produção do texto cifrado, principalmente nas situações de confito de posições.Para isso, após a implementação dos algoritmos, serão realizados testes para medir o desempenho do algoritmo.
	
	Também serão feitos testes de desempenho em outros algoritmos de cifra de fluxo e análises serão realizadas comparando o desempenho dos mesmos e do algoritmo proposto.
	
	\subsection{Melhorias na Implementação}
	\label{improments}
	
	Espera-se obter um desempenho inferior quando comparado a outros algoritmos de cifra de fluxo, porém se o mesmo for muito discrepante entre o algoritmo proposto e os outros algoritmos, melhorias serão avaliadas e se julgadas necessárias, serão implementadas.
	