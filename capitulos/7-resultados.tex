\chapter{Resultados}
\label{results}

Neste trabalho foram desenvolvidos diversos experimentos para se chegar a um resultado considerado satisfatório para a implementação do algoritmo proposto.O critério utilizado para a avaliação e validação dos experimentos foi o tempo total de execução. O tempo total de execução é a soma do processo da cifração combinada com a \textit{thread} de envio e a decifração. 

Os primeiros experimentos analisaram os tempos do processo de cifração e \textit{thread} de envio e a medição dos tempos médios necessários para a resolução dos conflitos.

Depois de alguns experimentos com os geradores: linear congruente, \textit{Blum Blum Shub} e \textit{Blum Micali}, chegou-se a conclusão que a utilização destes geradores complicaria a análise de desempenho entre o \textit{RC4} e o algoritmo aqui proposto. Isso se deve ao fato de que esses geradores são usados principalmente para geração de números, sem a preocupação com a sua repetibilidade. Para conseguir números com uma repetibilidade menor, seria necessário estudar o uso da chave fornecida para a cifração e transformá-la em entradas satisfatórias para estes geradores. Estes geradores demandam valores muito específicos para que o algoritmo tenha resultados satisfatórios e se esses requisitos não forem satisfeitos pode-se chegar a um \textit{loop} infinito na procura de um \textit{byte} que satisfaça as condições do algoritmo de balanceamento proposto.

O \textit{RC4}, por outro lado, se demonstrou bastante eficaz na geração de números esparsos o suficiente para minimizar os conflitos e evitar a possibilidade do \textit{loop} infinito, sendo, por esta razão, o único utilizado para este estudo.

Uma vez escolhida a melhor implementação do algoritmo, usou-se o \textit{RC4} puro para servir de base de tempo para verificar o \textit{overhead} imposto na cifrarão pelo algoritmo proposto.


\section{Ambiente Utilizado para Experimentos}

Foram utilizados dois computadores para os experimentos aqui descritos. O primeiro computador, que foi o responsável pela cifração do texto em claro, tem processador Intel i7, 16GB de memória RAM e utiliza o sistema operacional Ubuntu 14.04. O segundo computador, responsável pela decifração, tem processador Intel i7, 8GB de memória RAM e também utiliza Ubuntu 14.04 como sistema operacional. O compilador utilizado para ambos os computadores foi o \textit{gcc} 4.8.2.

Durante os experimentos, cabos de rede ligados diretamente no \textit{switch} de rede foram utilizados para que interferências da rede \textit{wireless} não prejudicassem os resultados dos experimentos. 

Como existem fatores externos aos experimentos(internet, por exemplo) os resultados tendem a variar alguns milissegundos. Para se obter o valor mais acurado, foram coletados dez resultados por tamanho de texto para cada algoritmo em cada experimento e calculada a média, que será apresentada nas tabela de resultados dos experimentos

\section{Tempos de Execução}

Esses experimentos foram realizados principalmente para a aproximação de um resultado satisfatório. O fato de se ter tratamento de conflitos faz com que o algoritmo fique com o desempenho inferior ao do \textit{RC4}, porém alguns ajustes foram feitos na arquitetura que foi planejada para a implementação do algoritmo para um resultado mais próximo possível.

\subsection{Experimento 1}

O primeiro experimento foi realizado utilizando somente a arquitetura que foi explanada no capítulo \ref{implementation-algorithm}, sem qualquer intervenção sobre o algoritmo. Os resultados foram totalmente insatisfatórios, como podem ser observados na tabela \ref{results-experiment-1}.

\begin{table}[h]
\centering
\begin{tabular}{|c|c|c|c|c|}
\hline
\multirow{2}{3cm}{Tamanho do texto(caracteres)} & \multicolumn{2}{c|}{RC4(ms)}     & \multicolumn{2}{c|}{RC4 + algoritmo(ms)} \\ \cline{2-5} 
                                     & Cifração & Cifração + \textit{Thread} & Cifração     & Cifração + \textit{Thread}     \\ \hline
10000                                & 12.3     & 13.4              & 18.75        & 19.75                 \\ \hline
50000                                & 51.6     & 54.2              & 97.875       & 98.875                \\ \hline
100000                               & 95.3     & 96.3              & 172.125      & 173.125               \\ \hline
500000                               & 1779.2   & 1836.6            & 3737.75      & 3740.5                \\ \hline
1000000                              & 6890.3   & 7023.6            & 9318.5       & 9325.75               \\ \hline
5000000                              & 48861.6  & 48905.4           & 54068        & 54083.625             \\ \hline
\end{tabular}
\caption{Tempos obtidos no experimento 1}
\label{results-experiment-1}
\end{table}

\subsection{Experimento 2}

Depois de análise dos resultados do primeiro experimento e a comparação dos resultados do \textit{RC4} e \textit{RC4} em conjunto com o algoritmo proposto, percebeu-se que algo que estava sendo utilizado na arquitetura do algoritmo e que não era o algoritmo em si estava afetando o desempenho da execução. A constatação se  se deve ao fato de que o \textit{RC4} também ter obtido resultados insatisfatórios. Depois de uma análise, descobriu-se que o tamanho da fila de mensagem \textit{default}, que estava sendo utilizado, era o problema. Devido ao seu tamanho estar limitado a poucos caracteres, a \textit{thread} de envio não estava sendo suficientemente rápida  e isso estava fazendo com que o tempo da cifração também fosse prejudicado. Outro indício que esse era o problema e não o algoritmo em si é que o tempo de cifração e tempo da \textit{thread} de envio era praticamente o mesmo. 

Com isso, foi efetuado o aumento da fila de mensagem e vimos que com a arquitetura implementada nesse trabalho isso é de extrema importância para o bom desempenho do algoritmo. Os resultados com a fila de mensagem expandida está na tabela \ref{results-experiment-2}.

\begin{table}[h]
\begin{tabular}{|c|c|c|c|c|}
\hline
\multirow{2}{3cm}{Tamanho do texto(caracteres)} & \multicolumn{2}{c|}{RC4(ms)}     & \multicolumn{2}{c|}{RC4 + algoritmo(ms)} \\ \cline{2-5} 
                                     & Cifração & Cifração + Thread & Cifração     & Cifração + Thread     \\ \hline
10000                                & 6.2      & 10.4              & 11.8         & 17.4                  \\ \hline
50000                                & 22.6     & 34.4              & 40           & 60.6                  \\ \hline
100000                               & 36.2     & 60                & 73           & 114.2                 \\ \hline
500000                               & 154.4    & 279.6             & 333.6        & 548.2                 \\ \hline
1000000                              & 304.2    & 544.2             & 663.2        & 1169.6                \\ \hline
5000000                              & 1400.4   & 3528              & 3139         & 6743.2      \\ \hline
\end{tabular}
\caption{Tempos obtidos no experimento 2}
\label{results-experiment-2}
\end{table}

\subsection{Experimento 3}

Com os resultados obtidos pelo experimento anterior, resolvemos reavaliar a necessidade de enviar \textit{byte} a \textit{byte} o texto cifrado. Isso porque depois de algumas análises, percebemos que a quantidade de conflitos em uma cifração era equivalente a metade do texto em claro e para cada conflito três \textit{bytes} a mais eram enviados, ou seja, a quantidade de bytes a mais enviados pelo algoritmo proposto era:

\begin{equation}
	tcifrado = tclaro + 2 \times (tclaro/2)
\end{equation}

Isso resulta no dobro do texto em claro e que somente o algoritmo proposto envia. Portanto, para diminuir a quantidade de envios decidimos enviar os três \textit{bytes} referentes ao conflito(sinal, quantidade de \textit{bytes} descartados e \textit{byte} cifrado) de uma só vez , aproximando, com isso, da quantidade de envios que o \textit{RC4} realiza. O resultado desse experimento está descrito na tabela \ref{results-experiment-3}.

\begin{table}[h]
\centering
\begin{tabular}{|c|c|c|c|c|}
\hline
\multirow{2}{3cm}{Tamanho do texto(caracteres)} & \multicolumn{2}{c|}{RC4(ms)}         & \multicolumn{2}{c|}{RC4+Algoritmo(ms)} \\ \cline{2-5} 
                                              & Cifração & Cifração + \textit{Thread} & Cifração  & Cifração + \textit{Thread}   \\ \hline
10.000                                        & 7.8      & 15.1                  & 12.6      & 15.9                   \\ \hline
50.000                                        & 19.3     & 33.6                  & 28.2      & 34.9                   \\ \hline
100.000                                       & 35.9     & 60.1                  & 50.5      & 61                     \\ \hline
500.000                                       & 157      & 279.2                 & 207.2     & 427.5                  \\ \hline
1.000.000                                     & 302.8    & 548.6                 & 397.6     & 1020.4                 \\ \hline
\end{tabular}
\caption{Tempos obtidos no experimento 3}
\label{results-experiment-3}
\end{table}

\subsubsection{Tempo Total de Execução}

Como os resultados anteriores foram os mais satisfatórios que obtivemos durante toda a fase de testes desse trabalho, decidimos dar uma visão de tempo de execução total. Nessa parte do experimento, o tempo de execução que foi levado em conta foram: cifração, \textit{thread} de envio e recebimento e decifração. Para se obter o tempo total foi necessário a utilização de um meio de informar para o lado da decifração que o lado da cifração havia acabado,porque senão a decifração ficaria esperando novos \textit{bytes} para decifrar. Portanto duas opções foram testadas:

\begin{description}
	\item [Envio de dois \textit{bytes}] Nesse experimento, para cada \textit{byte} cifrado enviado era enviado um \textit{byte} que indicava o final do texto em claro para o lado da decifração.
	
	\begin{table}[h]
\begin{tabular}{|c|c|c|c|c|}
\hline
\multirow{2}{3cm}{Tamanho do texto(caracteres)} & \multicolumn{2}{c|}{RC4(ms)}     & \multicolumn{2}{c|}{RC4 + algoritmo(ms)} \\ \cline{2-5} 
                                     & Cifração(ms) & Cifração + \textit{Thread} & Cifração     & Cifração + \textit{Thread}    \\ \hline
10000                                & 8.2      & 11.6              & 10.4         & 17.6                  \\ \hline
50000                                & 21.2     & 33.4              & 27           & 61.8                  \\ \hline
100000                               & 36.2     & 61.8              & 47.4         & 426.8                 \\ \hline
500000                               & 159.6    & 281.2             & 212.4        & 426.8                 \\ \hline
1000000                              & 311.2    & 574.4             & 419          & 835                   \\ \hline
5000000                              & 1540.8   & 2794.2            & 2086         & 4501.4                \\ \hline
\end{tabular}
\caption{Experimento tempo total enviando 2 \textit{bytes}}
\end{table}

Essa opção de envio foi descartada, visto que o tempo que a \textit{thread} utiliza para enviar aumentou consideravelmente. Portanto, decidimos pela estratégia que mostraremos a seguir.
	
	\item[Envio de sequência de caracteres] Nesse experimento, ao final do texto em claro foi enviado uma sequência de três bytes 0 como um marcador de fim de cifragem.
	
\begin{table}[h]
\centering
\begin{tabular}{|p{3cm}|c|c|c|}
\hline
Tamanho do texto(caracteres) & RC4(ms)  & RC4+Algoritmo(ms) & Tempo a mais(ms) \\ \hline
\centering 10.000                       & 125.2    & 127.6             & 2.4              \\ \hline
\centering 50.000                       & 605.9    & 623               & 17.1             \\ \hline
\centering 100.000                      & 1144.8   & 1214.6            & 69.8             \\ \hline
\centering 500.000                      & 5730     & 6042.2            & 312.2            \\ \hline
\centering 1.000.000                    & 11,429.6 & 12,048.7          & 619.1            \\ \hline
\end{tabular}
\caption{Tempo enviando sinal de fim do arquivo '000'}
\end{table}
\end{description}

Com esses resultados obtivemos em média 4,34\% de tempo a mais. A maior porcentagem obtida foi com o texto de tamanho 100.000 caracteres, 6,09\%.

\subsection{Resolução de Conflitos}

Esse experimento visa dar maior visibilidade sobre a quantidade média de conflitos que ocorrem em diversos tamanhos de textos e o tempo médio das resoluções. Esses tempos podem ter uma variação, visto que para cada conflito a função de obter o tempo do sistema foi utilizada duas vezes. Para textos pequenos a variação pode ser pequena, porém a medida que o texto fica maior, esse tempo tende a aumentar também.(Rever isso aqui, será mesmo que afeta o tempo de conflito ou só o tempo total?). A tabela \ref{amount-conflicts} mostra os resultados obtidos.

\begin{table}[h]
\centering
\begin{tabular}{|p{3cm}|c|c|p{3cm}|}
\hline
Tamanho do texto(caracteres) & Tempo Extra(ms) & Quantidade de Conflitos & Tempo médio de resolução(ms) \\ \hline
\centering 10.000                        & 10.4            & 5.037                    \centering & 0.002065                     \\ \hline
\centering 50.000                       & 32.8            & 25.174                   \centering &  0.001303                     \\ \hline
\centering 100.000                      & 67              & 49.998                   \centering & 0.001340                     \\ \hline
\centering 500.000                      & 313.7           & 250.086                  \centering & 0.001254                     \\ \hline
\centering 1.000.000                    & 611.8           & 499.890                  \centering & 0.001224                     \\ \hline
\centering 5.000.000                    & 3029.1          & 2.497.605                 & {0.001213}                     \\ \hline
\end{tabular}
\caption{Quantidade de conflitos e tempo médio de resolução.}
\label{amount-conflicts}
\end{table}

Percebe-se que a média de conflitos para cada tamanho de texto é aproximadamente a metade do tamanho do texto em claro. Outra percepção que pode ser obtida pela tabela é que conforme a quantidade de conflitos aumenta, o tempo médio de resolução diminui.