\chapter{Introdução}
\label{introduction}

Desde tempos anciões, segredos sempre existiram e métodos para deixa-los indecifráveis foram construídos. Enquanto algumas pessoas estavam escondendo informações, outras curiosos estavam tentando quebra-los. ~\cite{alexander-maximov}

Na constante luta entre manter e quebrar segredos, métodos foram desenvolvidos para se esconder informações e falhas nesses métodos foram exploradas para se obter as informações.

Um dos métodos para obter informações sem a autorização necessária é a análise de frequência dessas informações escondidas. Com essa análise, é possível se obter as informações. Há estudos que indicam a frequência média em que cada letra é utilizada em um texto e com isso é possível ter acesso a mensagem que deveria ser um segredo. 

\section{Objetivos do Trabalho}
\label{paper-objectives}

\subsection{Objetivo Geral}
\label{general-objective}
Esse trabalho visa apresentar um algoritmo para dificultar a análise de frequência de um texto e  assim se comunicar seguramente e privadamente.

\subsection{Objetivos Específicos}
\label{specifics-objectives}

\begin{itemize}
	\item Identificar e estudar as principais formas de criptografia.
	\item Identificar e estudar as principais formas de criptoanálise.
	\item Identificar e estudar geradores de números pseudo-aleatórios.
	\item Realizar a primeira proposta de algoritmo.
\end{itemize}

\section{Organização do Trabalho}
\label{paper-organization}

O trabalho tem a seguinte organização: no capítulo 2\ref{cryptography} é apresentado os principais tipos de algoritmos criptográficos incluindo a descrição de exemplo dos mesmos e também é descrito métodos de criptoanálise.

No capitulo 3\ref{prng} é descrito o que é um gerador de números pseudo-aleatórios, alguns exemplos e suas implementações e é explicado a importância desses geradores para a criptografia. 

No capitulo 4\ref{proposed-algorithm} é explicado a proposta do algoritmo com suas soluções, explicação do seu funcionamento e as principais vantagens em relação a outros algoritmos.

No capitulo 5\ref{conclusion} a conclusão desse trabalho é apresentada e a descrição de trabalhos futuros e suas importâncias são descritas. 