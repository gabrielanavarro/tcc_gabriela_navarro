\chapter{Introdução}
\label{introduction}

Desde tempos remotos, segredos sempre existiram e métodos para deixá-los indecifráveis foram construídos. Enquanto algumas pessoas estavam escondendo informações, outras curiosas estavam tentando descobri-los. ~\cite{alexander-maximov} %ok

Na constante luta entre manter e quebrar segredos, métodos foram desenvolvidos para se esconder informações e falhas nesses métodos foram exploradas para se obter as informações. 

Um dos métodos para obter informações sem a autorização necessária é a análise de frequência da ocorrência de caracteres dessas informações escondidas. Há estudos que indicam qual a frequência média de utilização de cada letra em um texto. Analisando um texto cifrado e calculando a frequência de ocorrência dos caracteres do mesmo, pode ser possível, através da substituição dos mesmos pelos caracteres com frequência similar da língua, quebrar a cifragem utilizada e ter acesso à mensagem original.

\section{Problema}

Como há a possibilidade de se obter as informações simplesmente analisando a frequência de ocorrência de caracteres, isso torna a comunicação insegura. Como consequência, o principal objetivo de se utilizar o algoritmo criptográfico, que é proteger informações importantes, torna-se obsoleto.  

\section{Objetivos do Trabalho}
\label{paper-objectives}

\subsection{Objetivo Geral}
\label{general-objective}
Esse trabalho visa apresentar um algoritmo de criptografia que dificulte a análise de frequência de um texto permitindo assim a comunicação privada e segura.

\subsection{Objetivos Específicos}
\label{specifics-objectives}

\begin{itemize}
	\item Identificar e estudar as principais formas de criptografia.
	\item Identificar e estudar as principais formas de criptoanálise.
	\item Identificar e estudar geradores de números pseudo-aleatórios.
	\item Realizar a proposta de algoritmo.
	\item Implementar o algoritmo proposto.
	\item Realizar experimentos afim de compara-los com resultados de outros algoritmos conhecidos.
\end{itemize}

\section{Organização do Trabalho}
\label{paper-organization}

Este trabalho tem a seguinte organização:

No capítulo \ref{cryptograhy} são apresentados os principais tipos de algoritmos criptográficos, incluindo suas descrições e exemplos, assim como são descritos métodos de criptoanálise.

No capitulo \ref{pseudo-random-number-generator} é descrito o que é um gerador de números pseudo-aleatórios, alguns exemplos e suas implementações e é explicada a importância desses geradores para a criptografia. 

No capitulo \ref{algorithm-proposition} é explicada a proposta do algoritmo com suas soluções, seu funcionamento e as principais vantagens em relação a outros algoritmos.

No capítulo \ref{implementation} é explicado como foi feito a implementação com a arquitetura que foi utilizada e os passos para utilizar o algoritmo.

No capítulo \ref{results} é explicado os experimentos realizados e como foi feita a coleta de dados desses experimentos. 

No capitulo \ref{conclusion} é apresentada a conclusão deste trabalho, a descrição dos trabalhos futuros e suas importâncias.