\chapter{Criptografia}
\label{cryptograhy}

%
A necessidade de manter informações em sigilo impulsionou a evolução dos estudos da criptografia. O exato início da criptografia é incerto, mas na Renascença,assim como outros em muitos outros campos, o estudo da criptografia começou a seu aprofundado e suas técnicas armazenadas e ensinadas. ~\cite{donald-davies}

%
Uma das primeiras formas de criptografia utilizada foi a substituição. A substituição primeiramente era feita de forma fixa, ou seja, era feita uma tabela fixa em que as letras do alfabeto eram trocadas por outras letras. Leon Battista Alberti introduziu uma maneira diferente de fazer a substituição, criando o algoritmo \textit{polyalphabetic substitution}. O princípio desse algoritmo era a criação de várias tabelas de substituição e a utilização de uma chave para definir a ordem que cada tabela seria usada para cifrar cada letra do texto em claro. O conjunto de tabelas de substituição é chamado de \textit{Vigenere tableau}

%
Durante a II Guerra Mundial a criação de novas máquinas foi incentivada e com isso máquinas para manter informações em sigilo foram criadas. Duas máquinas dessa época merecem destaque, o \textit{Enigma} e o 	\textit{SZ40}. Após a guerra, a publicação do algoritmo \textit{DES} serviu de incentivo para cientistas do mundo aprofundarem suas pesquisas em novos algoritmos. Nos dias de hoje existem duas linhas de pensamento sobre a criptografia, a criptografia com chave simétrica e assimétrica. 


\section{Criptografia Assimétrica}
\label{assymmetric-cryptography}

%
A criptografia assimétrica é usada principalmente em dados pequenos. Um dos principais usos é a assinatura digital e funções de hash. Esse tipo de criptografia está associada a criptografia de chave pública. Nela é estabelecido que um par de chaves (chave pública e privada) façam a cifração e decifração da mensagem. 

%
O funcionamento é simples. Bob tem suas duas chaves e disponibiliza sua chave pública de forma que qualquer pessoa que tenha interesse em enviar uma mensagem criptografada utilize sua chave pública. Quando Bob recebe a mensagem cifrada, é realizada a decifração dessa mensagem utilizando sua chave privada.

%
Existem muitos algoritmos de criptografia assimétrica. O Elgamal tem sua principal força proveniente da dificuldade de se calcular um logaritmo discreto.  O Diffie-Hellman foram os visionários da chave pública, primeiramente definiram a idéia e algum tempo depois definiram um algoritmo que serve apenas para trocar uma chave secreta entre duas pessoas.  Na tabela abaixo descrevemos os principais usos de cada algoritmo. 

\subsection{Algoritmo RSA}
\label{algorithm-rsa}

Tem seu nome baseado nas iniciais dos cientistas do \textit{MIT} que o inventaram: \textit{Ron Rivest, Adi Shamir} e \textit{Len Adleman}. A geração das chaves desse algoritmo é feita utilizando números primos e sua força está associada ao fato que é fácil realizar a multiplicação de dois números primos grandes, mas é difícil a decomposição desse resultado. Portanto a chave é gerada da seguinte forma:

\begin{enumerate}
\item Dois números \textbf{p} e \textbf{q} são escolhidos de forma aleatória e são primos.
\item \textbf{p} $\neq$ \textbf{q}
\item \textbf{n} é composto pela multiplicação de \textbf{p} e \textbf{q}.
\item \textbf{e}, tal que mdc($\phi$(n, e)) = 1
\item \textbf{d}, (e $^ {(-1)}$) * (mod $\phi$(n)) 
\end{enumerate}

%
A chave pública "e" é usada para criptografar a mensagem e a chave privada "d" é usada para descriptografar. O método de crifra é simples, o texto é tratado como conjunto de inteiros e a cada inteiro é feito o seguinte cálculo:

\begin{description}
\item [Criptografar]
C = M $^ e$ mod n. Sendo que M é o inteiro correspondente a parte da mensagem em claro e n o resultado da multiplicação dos dois números primos selecionados na confecção  das chaves, C é a mensagem cifrada e \textbf{e} é a chave pública. 
\item [Descriptografar]
M = C $^ d$ mod n. Sendo que C é o texto criptografado, M a mensagem em claro e \textbf{d} é a chave privada.
\end{description}

%
\section{Criptografia Simétrica}
\label{symmetric-cryptography}

A criptografia simétrica utiliza a mesma chave para criptografar e descriptografar uma mensagem. É objeto de estudo para muitos cientistas e existem muitos algoritmos que ainda são utilizados pelo mundo. 

Uma das vantagens em relação a criptografia assimétrica é que o custo para ser utilizada não é tão elevado e sua possibilidade de uso é maior, visto que a criptografia assimétrica tem suas limitações quanto ao tamanho da mensagem e o fato de se ter que criar duas chaves para cada um dos interessados na mensagem. O s algoritmos são divididos em: cifra de bloco e de fluxo.

\subsection{Cifra de Bloco}
\label{block-cipher}

A cifra de bloco tem como princípio o encriptamento de blocos com tamanhos definidos, sendo que o a mensagem criptografada deve ter o mesmo tamanho da mensagem em claro. Um algoritmo muito conhecido é o \textit{DES}. Desenvolvido pela \textit{IBM}, foi a primeira cifra comercialmente desenvolvida e que foi publicada por completo. ~\cite{achar-citação}. 

A maior fraqueza do algoritmo \textit{DES} é o tamanho da chave que é utilizada. Muitos pesquisadores sabendo desse problema tentaram remediar o problema com novas maneiras criptográficas, um exemplo é o \textit{3-DES}, que utiliza duas chaves no processo, mas como consequência tem seu desempenho prejudicado. Outra medida para contornar o problema de ataques ao \textit{DES} foi a publicação de um concurso do \textit{NIST} para a escolha do novo padrão criptográfico. Assim surgiu o \textit{AES}, algoritmo que utiliza o sistema de permutação e substituição, tem o tamanho do bloco fixo de 128 bits e tamanho de chave variável entre 128, 192 ou 256 bits. Existem operações que podem ser aplicadas nas cifras de blocos para a produção de mensagens criptografadas, são elas:

\begin{description}
\item [ECB]consiste em encriptar os blocos de forma independente um do outro e com uma chave fixa.
\item[CBC] utiliza da operação de ou-exclusivo entre um vetor de inicialização( para o primeiro bloco) e a mensagem em claro e por fim um ou-exclusivo do resultado com a chave. A partir do segundo bloco ao invés de um vetor de inicialização é utilizado o bloco criptografado anterior.
\item[CFB] muito similar ao CBC, porém é feito um ou-exclusivo da chave com o vetor de inicialização ou bloco criptografado anterior e o resultado é feito um ou-exclusivo com a mensagem em claro.
\item[OFB] a diferença entre o CFB é que o resultado entre o vetor de inicialização ou resultado da operação anterior e a chave serve de entrada para o próximo bloco, ao invés do bloco criptografado.
\item[CTR] utiliza como entrada para a função criptográfica um contador que é acrescentado de um em um. 
\end{description}

\subsection{Cifra de Fluxo}
\label{stream-cipher}

Os algoritmos que utilizam a cifra de fluxo o fazem de bit a bit ou byte a byte. Uma das vantagens da cifra de fluxo em relação a cifra de bloco é desempenho superior e por esse motivo é muito utilizado em sistemas de redes, como \textit{bluetooth}, \textit{SSL} e outros. Outra vantagem é a possibilidade de transformar uma cifra de bloco em cifra de fluxo, para isso basta definir o tamanho do bloco para ser bit ou byte. Alguns dos algoritmos de fluxo mais populares que hoje cobrem mais de 80$\%$ do mundo na telecomunicação e \textit{cyber space} são: A5/1, A5/2, E0 e RC4. ~\cite{pegar-citação}


\subsubsection{Algoritmo A5/1}
\label{algorithm-a51}

\subsubsection{Algoritmo A5/2}
\label{algorithm-a52}

\subsubsection{Algoritmo E0}
\label{algorithm-e0}

\subsubsection{Algoritmo RC4}
\label{algorithm-rc4}

\section{Criptoanálise}